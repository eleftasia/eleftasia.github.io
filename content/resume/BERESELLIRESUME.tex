\documentclass{resume} % Use the custom resume.cls style
\usepackage[utf8]{inputenc} % usually not needed (loaded by default)
\usepackage[T1]{fontenc}
\usepackage[left=0.2in,top=0.2in,right=0.2in,bottom=0.2in]{geometry} % Document margins
\newcommand{\tab}[1]{\hspace{.1\textwidth}\rlap{#1}} 
\newcommand{\itab}[1]{\hspace{0em}\rlap{#1}}
% \name{Eleftheria Beres | ellifteria@gmail.com} % Your name
\name{Eleftheria Beres} % Your name
% You can merge both of these into a single line, if you do not have a website.
\address{+1(817) 653-5991 || \href{mailto:ellifteria@gmail.com}{ellifteria@gmail.com} || \href{https://github.com/ellifteria}{github.com/ellifteria} || \href{https://github.com/ellifteria}{elliberes.me}}

\begin{document}

%----------------------------------------------------------------------------------------
%	EDUCATION SECTION
%----------------------------------------------------------------------------------------

\begin{rSection}{Education}

{\bf Northwestern University}, Evanston, IL \hfill Expected Graduation Date: Jun 2024\\
Bachelor of Science: Computer Science \hfill Cumulative GPA: 3.99

\end{rSection}

%----------------------------------------------------------------------------------------
%	RESEARCH EXPERIENCE SECTION
%----------------------------------------------------------------------------------------

\begin{rSection}{Research Experience}
    
    \textbf{Undergraduate Researcher: Xenobot Lab} \hfill Feb 2023- \\
     Northwestern University Center for Robotics and Biosystems \hfill \textit{Evanston, IL}
     \begin{itemize}
        \itemsep -3pt {} 
            \item[] \textbf{Project Maia}
            Designed and developed platform for simulating rigid-body virtual robots with the ability to grow as they behave in surroundings.
            Built tool allowing researchers to explore how evolutionarily optimized growth impacts the ability of simulated robots to learn and perform various behaviors.
            Presented work to CS Department at Summer Undergraduate Research Symposium.
            \item[] \textbf{ESRoCKit}
            Developed Julia and Python libraries to help build out the simulated robotics software ecosystem.
            Wrote Python library to control simulated robots using neural networks and Julia library to create robot definition files for physics simulators.
     \end{itemize}
     
     \textbf{Undergraduate Researcher: Leonard Lab} \hfill Dec 2021-\\
     Northwestern University Center for Synthetic Biology \hfill \textit{Evanston, IL}
     \begin{itemize}
        \itemsep -3pt {}
            \item[] \textbf{PyFlowBAT: An Open-Source Python Package for Flow Cytometry Batch Analysis}
            Conceptualized and developed Python package for easy-to-use, rapid flow cytometry data analysis for synthetic biologists.
            Collaborated iteratively with Ph.D. students and postdoctoral researchers at Northwestern to add features, ensure accurate results, and improve library accessibility and usability for non-computer scientists.
            Presented project project goals and progress in Leonard Lab group meetings
            Lead publication writing for PyFlowBAT paper—paper writing in progress, targeting early 2024 submission.
            Presented poster on PyFlowBAT at the Engineering Biology Research Consortium Annual Meeting 2023.
     \end{itemize}

     % \textbf{COMET Across Contexts} \hfill Dec 2021-Mar 2023\\
     % Northwestern University Leonard Lab \hfill \textit{Evanston, IL}
     % \begin{itemize}
     %    \itemsep -3pt {} 
     %        \item Experimentally characterize the performance of COMET synthetic transcription factors in novel contexts.
     %        \item Present project progress and discoveries in lab meetings.
     %        \item Collaborate with Ph.D. students on publication planning and writing.
     % \end{itemize}

\end{rSection} 

%----------------------------------------------------------------------------------------
%	PUBLICATIONS SECTION
%----------------------------------------------------------------------------------------

\begin{rSection}{PUBLICATIONS}

% \begin{rSection}{Poster Presentations}

\textbf{\textit{Poster Presentations}}

\begin{enumerate}
    \item \href{https://raw.githubusercontent.com/ellifteria/ellifteria.github.io/main/static/images/linked/ebrc-2023-poster.png}{\textbf{Beres E}, et al. PyFlowBAT: An Open-Source Software Package for Performing High-Throughput Batch Analysis of Flow Cytometry Data. Poster presented at: EBRC Annual Meeting; 2023 Jun 5-6; Evanston, IL.}
\end{enumerate}

\textbf{\textit{Manuscripts in Preparation}}

\begin{enumerate}
    \item Beres, E., Dreyer, K., Edelstein, H., Dray, K., \& Leonard, J. PyFlowBAT: An Open-Source Software Package for Performing High-Throughput Batch Analysis of Flow Cytometry Data. Manuscript in preparation.
    
    \item Dray, K., Edelstein H., Bora, G., Draut J., Kotzbauer, E., Beres, E., Muldoon, J., Lim, B., Schreiber, Y., Shah, P., Feng, S., Chen, A., Bagheri, N., Donahue, P., \& Leonard, J. Context-aware design of genetic programs. Manuscript in preparation.
    
\end{enumerate}

% \end{rSection}

\end{rSection} 

%----------------------------------------------------------------------------------------
%	HONORS SECTION
%----------------------------------------------------------------------------------------

\begin{rSection}{Honors and Awards}

\textbf{Summer Undergraduate Research Fellowship} \hfill Summer 2023 \\
     Northwestern University Department of Computer Science \hfill \textit{Evanston, IL}
     
\textbf{Summer Undergraduate Research Grant} \hfill Summer 2022, Summer 2023 \\
     Northwestern University Office of Undergraduate Research \hfill \textit{Evanston, IL}
    

\end{rSection}

%----------------------------------------------------------------------------------------
%	TEACHING EXPERIENCE SECTION
%----------------------------------------------------------------------------------------

\begin{rSection}{Teaching}

\textbf{Peer Mentor: GEN\_ENG 205-1: Engineering Analysis 1} \hfill Sept 2023-\\
Northwestern University Department of Electrical Engineering\hfill \textit{Evanston, IL}
 \begin{itemize}
    \itemsep -3pt {}
        \item[] Lead weekly group discussions over MATLAB homework assignments focusing on teaching how to break engineering problems into algorithmic steps and translate those steps into MATLAB code.
        Teach students linear algebra foundations and basic computational modeling in engineering.
        Guide students in using documentation and basic MATLAB debugging.
 \end{itemize}
 
\textbf{Peer Mentor: BMD\_ENG 220: Introduction to Biostatistics} \hfill Sep 2022-Dec 2022, Sep 2023-\\
Northwestern University Department of Biomedical Engineering \hfill \textit{Evanston, IL}
 \begin{itemize}
    \itemsep -3pt {} 
    \item[] Answer student questions, clarify confusing probability topics, and guide students on using Python to perform statistical tests during two weekly office hours virtually and in person.
    Facilitate students' understanding of statistical methods with a specific focus on practical applications to biomedical engineering experiments.
    Support teaching team with using Python and Jupyter.
    Wrote guides for how to use the basics of Python and Jupyter Notebooks for biomedical engineering students.
 \end{itemize}

\textbf{Peer Mentor: DATA\_ENG 200: Foundations of Data Science} \hfill Jan 2023-Mar 2023\\
Northwestern University Department of Computer Science \hfill \textit{Evanston, IL}
 \begin{itemize}
    \itemsep -3pt {}
        \item[] Answered student questions and facilitated learning of data science and data engineering in Python during three weekly office hours and on online course platform.
        Helped teach students data analysis, visualization, and computational problem-solving.
 \end{itemize}

\end{rSection} 


%----------------------------------------------------------------------------------------
% EXTRACURRICULAR
%----------------------------------------------------------------------------------------

%\begin{rSection}{Organizations} 
%
%    \textbf{Executive Board Member} \hfill Sep 2021-Mar 2023\\
%    Northwestern University Biomedical Engineering Society (BMES)\hfill \textit{Evanston, IL}
%     \begin{itemize}
%        \itemsep -3pt {}
%            \item \textbf{2022 - 2023: Secretary:} Communicate and coordinate BMES and Biomedical Engineering (BME) Department events with BME community; write and send biweekly BMES newsletter.
%            \item \textbf{2021 - 2022: Public Relations Chair:} Managed Northwestern BMES social media channels; publicized BME Department events.
%     \end{itemize}
%    
%\end{rSection}

%----------------------------------------------------------------------------------------
% SKILLS
%----------------------------------------------------------------------------------------

% \begin{rSection}{Skills and Interests}

%     \begin{tabular}{ @{} >{\bfseries}l @{\hspace{6ex}} l }
%     Skills: & Julia, C, Python, F\#, Git, Linux\\
%     Languages: & English (native), Greek (proficient), French (intermediate)\\
%     % Interests \& Passions: & Travel, Language learning, Climate and Social Justice, Speculative Fiction, World building\\
%     \end{tabular}\\

% \end{rSection}

\end{document}
